%\VignetteIndexEntry{PKgraph Overview}
%\VignetteDepends{PKgraph}
%\VignetteKeywords{PKgraph}
%\VignetteKeywords{PKgraph}
%\VignetteKeywords{PKgraph}
%\VignettePackage{PKgraph}
\documentclass[a4paper]{article}

\newcommand{\Rfunction}[1]{{\texttt{#1}}}
\newcommand{\Robject}[1]{{\texttt{#1}}}
\newcommand{\Rpackage}[1]{{\textit{#1}}}
\newcommand{\Rclass}[1]{{\textit{#1}}}
\newcommand{\Rmethod}[1]{{\textit{#1}}}

\author{Xiaoyong Sun$^\dagger$$^\ddagger$\footnote{johnsunx1@gmail.com}}

\usepackage{Sweave}
\usepackage{hyperref}
\begin{document}

\setkeys{Gin}{width=1\textwidth}

\title{User Guide for PKgraph Package}
\maketitle
\begin{center}$^\dagger$Binformatics and Computational Biology Program, $^\ddagger$Department of Statistics \\ Iowa State University, Ames, Iowa 50010, USA
\end{center}

\tableofcontents
%%%%%%%%%%%%%%%%%%%%%%%%%%%%%%%%%%%%%%%%%%%%%%

\section{Introduction}
Population pharmacokinetic (PopPK) modeling has become increasing important in
drug development because it allows unbalanced design, sparse data and the study
of individual variation. However, this complexity of the model makes it a challenge
to diagnose the fit.  Graphics can play an important and unique role in PopPK model diagnostics.
The software described in this paper, PKgraph, provides a graphical user interface for
PopPK model diagnosis with interactive graphics. It also provides an integrated and comprehensive platform for analysis
of pharmacokinetic data including exploratory data analysis, goodness of model fit,
model validation and model comparison. It can be used with a variety of modeling fitting software,
including NONMEM, Monolix, SAS and R. PKgraph is programmed in R, and uses the R packages
lattice, ggplot2 for static graphics, and rggobi for interactive graphics. This R package is
supported with a user-friendly graphical user interface so that users can easily control diagnosing
with simple clicks. The PKgraph software serves as a supplement to the existing packages:
NONMEM, Xpose and PsN for diagnosing models.
\newline
\newline
PKgraph is an R packaged built on the following R packages: RGtk2, gWidgets, gWidgetsRGtk2,
lattice, and ggplot2. It requires R ($>$ 2.0) and  GTK+, and runs under Windows, Linux and Mac.
\section{Installation}
PKgraph needs to install the following programs and R packages:
\newline \newline
1. install GTK \newline
For Windows, you can download the GTK Developer's Pack from
\newline
   http://gladewin32.sourceforge.net/
\newline
\newline
For Unix, you can fetch the source files for the different
libraries from
\newline
   ftp://ftp.gtk.org/pub/gtk/v2.8/
\newline
\newline
2. Install RGtk2 (Please see RGtk2 Installation notes if you have problems) \newline
\textit{install.packages(``RGtk2'')} \newline \newline
3. install rggobi   \newline
a. Download and install ggobi (www.ggobi.org)  \newline
b. Install rggobi: \textit{install.packages(``rggobi'')} \newline \newline
4. Install gWidgets  \newline
\textit{install.packages(``gWidgets'')} \newline \newline
5. Install cairoDevice \newline
\textit{install.packages(``cairoDevice'')} \newline \newline
6. Install gWidgetsRGtk2 \newline
\textit{install.packages(``gWidgetsRGtk2'')} \newline \newline
7. Install lattice  \newline
\textit{install.packages(``lattice'')} \newline \newline
8. Install ggplot2  \newline
\textit{install.packages(``ggplot2'')} \newline \newline
\section{User Guide}
Please see the user guide at \url{https://sourceforge.net/projects/pkgraph/files/}



\end{document}
